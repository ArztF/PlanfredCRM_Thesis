Dadurch, dass die Cron-Jobs nur einmal am Tag ausgeführt werden, müssen die Daten dementsprechend zwischengespeichert werden, um dauerhaft dem Supportteam zur Verfügung zu stehen. Dazu wurden folgende Collections erstellt:
\begin{itemize}
    \item \textbf{Statistics}
        \newline
        In dieser Collection werden die Anzahl an Benutzer, Projektbesitzer, hochgeladenen Plane und alle weiteren relevanten Statistiken, welche auf der Startseite angezeigt werden gespeichert.
    \item \textbf{ProjectOwners}
        \newline
        In dieser Collection werden alle IDs der Benutzer gespeichert, welche ein eigenes Projekt besitzen (permission=owner)
    \item \textbf{SecretUsers}
        \newline
        In dieser Collection werden alle Benutzer und Projekte dieses Benutzer gespeichert, welche keinen Zahlungsplan aktiviert haben, ein eigenes aktives Projekt besitzen, in welchem in den letzten 10 Wochen ein Upload durchgeführt wurde. Dadurch wird verhindert, dass ein nicht zahlender Kunde, trotzdem ein Projekt verwaltet.
    \item \textbf{UsersWithExceededStorage}
        \newline
        In dieser Collection werden alle Benutzer gespeichert, welche ihren gebuchten Speicherplatz überschritten haben.
    \item \textbf{UsersWithoutProjectsAndParticipants}
        \newline
        In dieser Collection werden alle Benutzer gespeichert, welche sich ausschließlich einen Account erstellt haben, allerdings Planfred nicht aktiv nutzen.
    \item \textbf{ChangeLogs}
        \newline
        In diese Collection werden alle Write-Operationen des Supportteams dokumentiert, um diese auf einer eigenen Log-Seite anzuzeigen, um evtl. Fehler leichter zu finden und diese einfacher rückgängig zu machen.
\end{itemize}