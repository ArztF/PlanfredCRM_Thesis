Um Cron-Jobs in diesem express.js Projekt zu implementieren muss man lediglich das NPM-Package \textbf{node-cron} installieren.
\begin{verbatim}
npm install node-cron
\end{verbatim}
Nach Abschluss der Installation wurde ein eigenes JavaScript File erstellt, in welchem das Cron-Modul importiert wird.
\begin{verbatim}
const cron = require('node-cron')
\end{verbatim}
Mit der oben erstellen cron-Variable ist es nun möglich jegliche Cron-Jobs einzuplanen.
\newline
\begin{lstlisting}
cron.schedule('0 22 * * *', <auszufuehrender Befehl>)
\end{lstlisting}
Dabei wurde im Zuge dieser Diplomarbeit ebenfalls noch ein if-Statement eingebaut, welches mittels .env Variablen überprüft, ob der Server gerade im sogenannten "production" Modus läuft. Falls das Backend also lokal gestartet wird, werden alle Cron-Jobs erneut ausgeführt, unabhängig von der eingeplanten Zeit. Im folgenden Codebeispiel sieht man die oben beschriebenen Funktionen.
\newline
\begin{lstlisting}
const cronJobs = async () => {
  // on localhost, run cronjobs immediately after startup
  if (process.env.NODE_ENV !== 'production' && !process.env.MONGODB_CONNECTION_STRING) {
    await estimateUsersWithExceededStorage()
    await estimateUsersWithoutProject()
    await estimateStatistics()
    await estimateProjectOwners()
    await estimateSecretUsers()
  }

  logger.info('Cron Jobs scheduled')
  cron.schedule('0 22 * * *', estimateUsersWithExceededStorage)
  cron.schedule('0 22 * * *', estimateStatistics)
  cron.schedule('50 21 * * *', estimateProjectOwners)
  cron.schedule('0 22 * * *', estimateSecretUsers)
}

module.exports.cronJobs = cronJobs
\end{lstlisting}
Die gesamte Funktion mit der Funktionalität wird schlussendlich exportiert und im server.js File wieder importiert. In diesem wird die Funktion nach dem erfolgreichen Verbinden zu den beiden Datenbanken aufgerufen und dadurch eingeplant.
