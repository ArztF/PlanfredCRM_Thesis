Cron-Jobs sind automatisierte Aufgaben, die in unixartigen Betriebssystemen, wie Linux, MacOS und Serverumgebungen ausgeführt werden können. Es gibt den sogenannten Cron-Daemon, welcher die Grundlagen für die Cron-Jobs bildet und im Hintergrund mitläuft. Dieser gibt die zeitlichen Impulse für die Aufgaben und führt diese zur angegebenen Zeit aus. Die Syntax, um einen Cron-Job an einer bestimmten Uhrzeit zu starten, ist stellt sich folgendermaßen dar:
\cite{cron_jobs_basics}
\newline
\begin{lstlisting}
'* * * * *', <auszufuehrender Befehl>
\end{lstlisting}
Der fünfte Stern steht für den Wochentag, an dem der Befehl ausgeführt werden soll. In diesem Fall sind Werte von 0 bis 7 möglich, wobei 0 und 7 für den Sonntag stehen.
\newline
Der vierte Stern steht für den Monat, hierbei sind anstelle des Sterns Zahlen von 1 bis 12 möglich.
\newline
Der dritte Stern steht für den Wochentag und es sind Zahlen von 1 bis 31 möglich. Wählt man die Zahl 31, wird dieser Befehl nur in Monaten ausgeführt, die 31 Tage haben. Es wird dementsprechend kein Zeitintervall gestartet und alle 31 Tage ausgeführt, sondern es wird tatsächlich vom Cron-Daemon der Tag überprüft.
\newline
Der zweite Stern gibt an zu welcher Stunde der Befehl ausgeführt werden soll. Dabei sind Zahlen von 0 bis 23 möglich.
\newline
Der erste Stern gibt die Minuten an und kann Zahlen von 0 bis 59 annehmen.
\newline
Es ist ebenfalls möglich, die Sterne nicht durch Zahlen zu ersetzen. In diesem Fall wird der Befehl jede Minute, jede Stunde, jeden Tag, jeden Monat oder jeden Wochentag ausgeführt. Möchte man den Befehl, zum Beispiel, jeden Tag um 8:00 Uhr und um 20:00 Uhr ausführen, sieht der Befehl wie folgt aus:
\newline
\begin{lstlisting}
'0 8,20 * * *', <auszufuehrender Befehl>
\end{lstlisting}
Möchte man zum Beispiel einen Befehl alle 10 Minuten ausführen, muss man nicht wie im obigen Beispiel alle Minuten manuell eingeben, sondern kann dies wie folgt realisieren:
\newline
\begin{lstlisting}
'*/10 * * * *', <auszufuehrender Befehl>
\end{lstlisting}
\cite{cron_jobs_scheduling_examples}