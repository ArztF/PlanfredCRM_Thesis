In diesem Endpunkt werden ausschließlich User Objekte an den Benutzer oder die Benutzerin zurückgegeben. Aufgrund von der Performance, werden bei allen Endpunkten nur die ersten 25 Objekte aus der Datenbank abgefragt. Mit einem Button im Frontend kann man die nachfolgenden 25 Benutzer nachladen. Folgende Endpunkte wurden dabei definiert und erledigen folgende Aufgaben:
\begin{itemize}
    \item \textbf{/users-all}
        \newline
        Bei diesem Endpunkt werden alle Benutzer aus der Datenbank zurückgeliefert. Es gibt die Optionen, die Benutzer nach Logindatum, Registrierungsdatum oder Anzahl der Projekte zu sortieren.
        \newpage
    \item \textbf{/users-search-email}
        \newline
        Bei diesem Endpunkt werden ebenfalls standardmäßig alle Benutzer abgefragt, es gibt jedoch die Möglichkeit nach einer E-Mail zu suchen. Dabei wird der eingegebene String ebenfalls als Übergabeparameter an das Backend übermittelt und dort überprüft, ob der String in der E-Mail enthalten ist.
    \item \textbf{/users-without-project}
        \newline
        Bei diesem Endpunkt werden alle Benutzer zurückgegeben, welche sich selbst registriert (eigene Entity in Collection) und kein eigenes Projekt haben (ebenfalls eigene Entity in Collection).
    \item \textbf{/users-storage-exceeded}
        \newline
        Bei diesem Endpunkt werden alle Benutzer ermittelt, welche mit ihren eigenen Projekten den gebuchten Speicherplatz überschritten haben.
    \item \textbf{/users-have-billing-plan}
        \newline
        In diesem Endpunkt gibt es zwei verschiedene Möglichkeiten die Benutzer anzeigen zu lassen. Einerseits kann man alle Benutzer anzeigen, die noch keinen Zahlungsplan ausgewählt haben. Andererseits ist es möglich alle Benutzer die einen Zahlungsplan ausgewählt haben anzuzeigen. Diese Unterscheidung wird ebenfalls mit einem Übergabeparameter gelöst.
    \item \textbf{/users-deleted}
        \newline
        Grundsätzlich ist wichtig zu erwähnen, dass, wenn ein Benutzer seinen Account eigenständig löscht, dieser nicht komplett aus der Datenbank entfernt wird, sondern eine Entity in der Collection auf true gesetzt wird. Deshalb werden bei diesem Endpunkt alle Benutzer angezeigt, welche ihren Account gelöscht haben.
    \item \textbf{/users-full-text-search}
        \newline
        Bei diesem Endpunkt werden alle Entitäten der Collection untersucht und auf einen Treffer geprüft und anschließend ans Frontend gesendet.
    \item \textbf{/users-search-id}
        \newline
        Hier wird die ID an das Backend übergeben und der User mit dieser zurückgegeben.
    \item \textbf{/users-domain-search}
        \newline
        Dies ist ein eigener Endpunkt, bei welchem nur auf die Domain der Email-Adresse auf einen Treffer überprüft wird.
    \item \textbf{/users-secret-users}
        \newline
        Bei diesem Endpunkt werden alle Benutzer zurückgeliefert, welche keinen Zahlungsplan ausgewählt haben, jedoch trotzdem ein eigenes Projekt besitzen, in welchem in den letzten 30 Tagen ein Upload durchgeführt worden ist.
    \item \textbf{/users-domain-plan-search}
        \newline
        Bei diesem Endpunkt können alle Benutzer zurückgegeben werden, die keinen Zahlungsplan aktiviert haben und einen bestimmten Teil in der E-Mail haben.
    \item \textbf{/changePlan/:id}
        \newline
        Bei diesem Endpunkt kann man den Zahlungsplans eines Nutzers auf none setzen, sodass kein Zahlungsplan mehr aktiv ist und der Benutzer nichts mehr bezahlen muss.
    \item \textbf{/editPlan/:id}
        \newline
        Bei diesem Endpunkt kann man den Zahlungsplan individuell verändern. Man kann zum Beispiel den Speicherplatz verändern oder einen anderen vordefinierten Zahlungsplan auswählen.
\end{itemize}