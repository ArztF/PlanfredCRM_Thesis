Der bedeutendste Unterschied zwischen den beiden Frameworks besteht darin, dass Express dem Entwickler die Freiheit gewährt, die Art und Weise der Code-Implementierung nach eigenem Ermessen zu gestalten, da keine vordefinierten Regeln vorgeschrieben sind. Im Gegensatz dazu legt Nest mit seinem Leitprinzip "Konvention vor Konfiguration" besonderen Wert darauf, eine vordefinierte Struktur und klare Konventionen zu etablieren. Dies kann in größeren Entwicklerteams von Vorteil sein, da es dazu beiträgt, einheitliche Regeln und Strukturen zu gewährleisten, denen alle Mitglieder folgen müssen.
\newline
Aufgrund dieser Überlegungen traf das vorliegende Projektteam die Entscheidung, Express.js statt Nest.js zu verwenden. Das Team besteht aus drei Mitgliedern, die in räumlicher Nähe arbeiten und daher in der Lage sind, die "Regeln" und Strukturen gemeinsam festzulegen, ohne auf vorgegebene Konventionen angewiesen zu sein. Eine weitere maßgebliche Überlegung für die Wahl von Express.js war das bereits vorhandene Know-how im Unternehmen. Alle bisherigen Anwendungen wurden mithilfe von Express.js entwickelt, was die Entscheidung für Express.js als das präferierte Framework begründete.
\cite{express_js_vs_nest_js}