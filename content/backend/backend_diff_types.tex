JavaScript ist immer noch für viele Entwicklerinnen und Entwickler überwiegend als Frontend Programmiersprache bekannt. 

"More than 97\% of websites rely on JavaScript, and all major web browser support it"
(https://blog.hubspot.com/website/java-backend; Zugriff: 26.10.2023)

Dieses Zitat könnte ein Grund dafür sein. Es wurde außerdem nie geplant, JavaScript als Backend-Programmiersprache zu erweitern. Der Hintergedanke bei der Entwicklung von JavaScript war, eine kompakte Scriptsprache zu veröffentlichen, um dynamische Web-Erlebnisse zu ermöglichen. Allerdings wurde JS, nicht wie andere Frontend-Webtechnologien, ausschließlich für visuelle Darstellungen, sondern wurde als vollwertige Scriptsprache mit erweiterten Funktionen entwickelt. Dies ist einer der Gründe, warum JavaScript nun auch als Server-Side-Script Sprache in der IT verwendet wird.
\newline
Der Vorteil eines JavaScript Backends ist, dass man sowohl im Frontend, als auch im Backend, auf eine gemeinsame Sprache setzen kann. Aus diesem Grund hat sich dieses Projektteam für ein auf JavaScript basierendes Backend entschieden.
\newline
Nachdem nun die Entscheidung für ein JavaScript Backend erläutert wurde, richtet sich nun das Augenmerk auf die zahlreichen verfügbaren Frameworks. In der folgenden Analyse werden die verschiedenen Optionen beleuchtet und ihre spezifischen Eigenschaften sowie ihre Anwendbarkeit im Kontext zu dieser Projektarbeit evaluiert.
\cite{Backend_JavaScript}