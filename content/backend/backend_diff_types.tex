JavaScript ist immer noch für viele Entwickler:innen überwiegend als frontend Programmiersprache bekannt. 

"More than 97\% of websites rely on JavaScript, and all major web browser support it"
(vgl. https://blog.hubspot.com/website/java-backend; Zugriff: 26.10.2023)

Diese Statistik könnte ein Grund dafür sein. Es wurde außerdem nie geplant JavaScript als backend Programmiersprache zu erweitern. Der Hintergedanke bei der Entwicklung von JavaScript, war es eine kompakte script Sprache zu veröffentlichen, um dynamische Web-Erlebnisse zu ermöglichen. Allerdings wurde JS, nicht wie andere Frontend-Webtechnologien, nicht ausschließlich für visuelle Darstellungen entwickelt, sondern wurde als vollwertige Scriptsprache mit erweiterten Funktionen entwickelt. Dies ist mit einer der Gründe, warum JavaScript nun auch als Server-Side-Script Sprache in der IT verwendet wird.
\newline
Der Vorteil eines JavaScript Backends ist, dass man sowohl im Frontend, als auch im Backend, sich auf eine Sprache verlassen kann. Dies ist der Hauptgrund, warum sich unser Projektteam für ein JavaScript basierendes Backend entschieden hat.
\newline
Nachdem nun die Entscheidung für ein JavaScript Backend erläutert wurde, richtet sich nun das Augenmerk auf die zahlreichen verfügbaren Frameworks. In der folgenden Analyse werden die verschiedenen Optionen beleuchtet und ihre spezifischen Eigenschaften sowie ihre Anwendbarkeit im Kontext zu dieser Projektarbeit evaluiert.
\cite{Backend_JavaScript}