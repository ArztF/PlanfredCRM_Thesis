Postmark ist ein Online-Service, welcher eine Email API zur Verfügung stellt. Diese API wurde in dieser Diplomarbeit verwendet um Mails an das Support Team zu versenden sobald eine Änderung in der Datenbank erfolgt ist. Das Hauptaugenmerk dieses Service liegt aber darin, dass die Emails auch wirklich in der Inbox ankommen und nicht im Spam landen. Diese Plattform implementiert bewährte Praktiken, um sicherzustellen, dass die Emails auch wirklich zuverlässig beim Empfänger ankommen. Dies funktioniert mit Funktionen wie Echtzeit-Protokollierung und Mechanismen zur Vermeidung von Spam.

Der Nachteil von Postmarkt liegt allerdings in den Kosten die dadurch verursacht wird. 10.000 Emails kosten monatlich 15 USD, bis zu einer Anzahl von 100 Emails pro Monat ist dieser Service kostenlos.

\begin{figure}[h!]
    \centering
    \includegraphics[width=0.3\textwidth]{pics/postmark.png}
    \caption{Postmark}
    \label{fig:enter-label}
\end{figure}


\subsection{Mailversandfunktion}

In der folgenden Diplomarbeit wurde Postmarkt wie folgt verwendet


\begin{lstlisting}
// Send an email:
const client = new postmark.ServerClient(config.postmark.apikey)

async function mail (sub, body) {
  client.sendEmail({
    From: `${config.postmark.from}`,
    To: 'support@planfred.com',
    Subject: `${sub}`,
    HtmlBody: `${body}`
  })
}
\end{lstlisting}

Zuerst wird ein Client angelegt:
\begin{lstlisting}
new postmark.ServerClient(config.postmark.apikey)
\end{lstlisting}
In der Config ist der API-Key definiert welcher benötigt wird um Postmark verwenden zu können.

Die Funktion "client.sendEmail()" braucht einige Parameter:

\begin{itemize}
    \item \textbf{From}
        \newline
        Hier wird die Absender E-Mail Adresse definiert. Diese kann beliebig gewählt werden.
    \item \textbf{To}
        \newline
        In diesem Parameter wird die Empfänger E-Mail definiert.
    \item \textbf{Subject}
        \newline
        Im Subject wird der Betreff des E-Mails definiert.
    \item \textbf{HtmlBody}
        \newline
        Im HtmlBody wird der Inhalt in die E-Mail geschrieben. Dieser kann als HTML Code geschrieben werden.
\end{itemize}


Weiteres können noch andere Properties wie zum Beispiel Header, TextBody, oder Cc gesetzt werden:

\begin{lstlisting}
{
  "From": "sender@example.com",
  "To": "receiver@example.com",
  "Cc": "copied@example.com",
  "Bcc": "blank-copied@example.com",
  "Subject": "Test",
  "Tag": "Invitation",
  "HtmlBody": "<b>Hello</b>",
  "TextBody": "Hello",
  "ReplyTo": "reply@example.com",
  "Metadata": {
      "Color":"blue",
      "Client-Id":"12345"
  },
  "Headers": [
    {
      "Name": "CUSTOM-HEADER",
      "Value": "value"
    }
  ],
  "TrackOpens": true,
  "TrackLinks": "HtmlOnly",
  "MessageStream": "outbound"
}
\end{lstlisting}


Mit der "mail" Funktion wird also eine Email versendet diese funktioniert einfach nur durch 
\begin{lstlisting}
client.sendEmail({})
\end{lstlisting}
