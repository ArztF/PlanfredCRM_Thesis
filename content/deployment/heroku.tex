Heroku ist eine weit verbreitete Plattform-as-a-Service-Lösung, die Entwicklern das einfache Bereitstellen, Skalieren und Verwalten von Anwendungen ermöglicht. Diese Plattform unterstützt eine Vielzahl von Programmiersprachen, darunter Java, Ruby, PHP, Node.js, Python, Scala und Clojure. Heroku führt Anwendungen in virtuellen Containern aus, die als Dynos bezeichnet werden.

Für die Nutzung von Heroku werden den Nutzern die virtuellen Maschinen in Rechnung gestellt, die für ihre Anwendungen benötigt werden. Die Plattform von Heroku und die von den Benutzern erstellten Anwendungen nutzen die Amazon Web Services als zugrunde liegende Infrastruktur. Dies ermöglicht Entwicklern eine schnelle Anwendungsentwicklung aufgrund der hohen Benutzerfreundlichkeit.

\subsection{Was ist ein Dyno?}

Ein Dyno ist ein Container auf der Heroku-Plattform, der für die Ausführung und Skalierung von Heroku-Anwendungen verwendet wird. Im Wesentlichen handelt es sich um virtuelle Linux-Container, in denen Code basierend auf Benutzerbefehlen ausgeführt wird.

Entwickler:innen haben die Möglichkeit, die Anzahl der Dynos je nach den Anforderungen ihrer Anwendungen zu skalieren. Heroku stellt Funktionen zur Containerverwaltung bereit, mit denen Benutzer die Größe, den Typ und die Anzahl der Dynos problemlos entsprechend den Bedürfnissen ihrer Anwendungen skalieren und verwalten können.

Dynos sind die grundlegenden Bausteine, die Heroku-Anwendungen antreiben. Entwickler:innen können ihre Applikationen auf Dynos bereitstellen und diese Einheiten verwalten, um einfach skalierbare Anwendungen zu erstellen und auszuführen.

Durch diese Bereitstellung werden Softwareentwickler:innen von der Infrastrukturverwaltung befreit und können sich stattdessen auf die wichtigen Aspekte der Anwendungsentwicklung und -ausführung konzentrieren.

\begin{figure}[h!]
    \centering
    \includegraphics[width=0.7\linewidth]{pics/heroku.png}
    \caption{Heroku Logo}
    \label{fig:enter-label}
\end{figure}
