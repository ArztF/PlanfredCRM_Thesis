Netlify wurde 2014 von Matt Biilmann und Chris bach gegründet und ist eine Cloud-Plattform, welche sich spezial um die Bereitstellung und das Hosting von Webanwendungen kümmert. Die Plattform entstand damals aus der Idee den Entwicklern den Prozess der Bereitstellung einer Webanwendung zu vereinfachen und wurde mittlerweile zu einer der beliebtesten der beliebtesten Hosting Plattformen.

Die Cloud-Plattform bietet einige sehr nützliche Features:

\subsubsection{Continuous Deployment}
Sobald ein Commit auf in einem unterstützen Versionskontrollsystem wie zum Beispiel Git passiert wird automatisch ein neues Deployment ausgelöst. Der Entwickler muss sich hierbei nicht mehr manuell um den Build- und Bereistellungsprozess kümmern.

\subsubsection{Branch Deployment}
Mit Netlify ist es möglich Testumgebungen zu erstellen mit einzelnen Branches. Jeder Branch kann in einer seperaten Umgebung bereitgestellt werden, so können Entwickler ihre Änderungen testen bevor sie in einen Produktionsbranch gepusht werden.

\subsubsection{Automatisches SSL-Zertifikat}
Über die Cloud-Plattform Netlify wird automatisch ein SSL/TLS-Zertifikat erstellt welches Sicherheit für die gehostete Website gewährleistet.

\subsubsection{Serverless Functions}
Netlify bietet Unterstützung für serverlose Funktionen, welche Entwicklern die Möglichkeit bieten, maßgeschneiderte Backend-Logik in ihren Projekten zu implementieren. Diese Funktionen skalieren automatisch nach Bedarf und eignen sich hervorragend für die Erstellung von APIs und anderen serverlosen Anwendungen.

\cite{Was_ist_Netlify}
