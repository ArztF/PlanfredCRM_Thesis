In der Softwareentwicklung gibt es mehrere verschiedene Arten von Tests, die in unterschiedlichen Phasen der Entwicklung eingesetzt werden.

\subsection{Unit-Tests}
Unit-Tests sind sehr einfach zu schreiben und testen die Anwendung nah an der Quelle. Mit ihnen werden die einzelnen Methoden und Funktionen der von der Software verwendeten Klassen, Komponenten oder Module getestet. Sie zielen darauf ab, dass jede Einheit des Codes isoliert und fehlerfrei arbeitet. In der Regel lassen sich Unit Tests auch sehr kostengünstig automatisieren, so können sie von einem Continuous-Integration-Server sehr schnell durchgeführt werden.


\subsection{Integrationstests}
Das Ziel von Integrationstests ist es nicht wie die Unit-Tests die einzelnen Komponenten auf ihre Funktionalität zu prüfen, Integrationstests prüfen ob die einzelnen Module oder Komponenten auch ordnungsgemäß zusammenarbeiten. So kann die Interaktion mit der Datenbank oder das Zusammenspiel von anderen Services getestet werden. Im Falle dieser Diplomarbeit könnte man zum Beispiel testen, ob Postmark auch mit der benutzten MongoDB Datenbank kompatibel ist.

\subsection{Funktionstests}
Diese Art von Tests, bekannt als Funktionstests, haben das Ziel, die Funktionalität der Anwendung als Ganzes zu überprüfen, im Gegensatz zu Unit-Tests, die einzelne Komponenten isoliert testen. Funktionstests sind darauf ausgerichtet sicherzustellen, dass alle Teile der Anwendung ordnungsgemäß zusammenarbeiten und miteinander kommunizieren können. Sie simulieren typische Benutzerinteraktionen und prüfen, ob die verschiedenen Komponenten korrekt integriert sind und das erwartete Verhalten zeigen, wenn sie gemeinsam ausgeführt werden.

\subsection{End-to-End-Tests}
E2E-Tests sind eine Art von Softwaretests, bei deinen ein großer Bereich der ganzen Anwendung getestet wird um sicherzustellen, dass die Anwendung auch in einer realen Umgebung ordnungsgemäß funktioniert. Der Sinn besteht darin, dafür zu sorgen, dass alle Komponenten und Systeme der Applikation wie erwartet funktionieren. Durch E2E-Tests können Probleme erkannt werden, die in isolierten Unit- oder Integrationstests unter Umständen unentdeckt bleiben. Diese Art von Tests sind sehr nützlich, aber auch kostspielig und in automatisierter Form eher schwierig zu verwalten. Für diese Diplomarbeit wurde Cypress als Test-Runner für die geschriebenen End-to-End-Tests verwendet.
\cite{Verschiedene_Testarten}
