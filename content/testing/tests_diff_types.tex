In der Softwareentwicklung gibt es mehrere verschiedene Arten von Tests, die in verschiedenen Phasen der Entwicklung eingesetzt werden.

\subsubsection{Unit-Tests}
Unit Tests überprüfen die einzelnen Komponenten oder Module der Software auf ihre Funktionalität. Sie zielen darauf ab, dass jede Einheit des Codes isoliert und fehlerfrei arbeitet.


\subsubsection{Integrationstests}
Das Ziel von Integrationstests ist es nicht wie die Unit-Tests die einzelnen Komponenten auf ihre Funktionalität zu prüfen, Integrationstests prüfen ob die einzelnen Module oder Komponenten auch ordnungsmeäß zusammenarbeiten.

\subsubsection{Funktionstests}
Diese Art von Tests sind darauf ausgerichtet, ähnlich wie die Unit-Tests die Funktionalität zu überprüfen, jedoch testen diese die Funktionalität der Anwendung alles ein Ganzes und nicht nur die einezlnen Komponenten.

\subsection{End-to-End-Tests}
E2E-Tests sind eine Art von Softwarestests, bei deinen ein großer Bereich der ganzen Anwendung getestet wird um sicherzustellen, dass die Anwendung auch in einer realen Umgebung ordnungsgemäß funktioniert. Der Sinn besteht darin, dass sichergestellt wird, dass alle Komponenten und Systeme der Applikation wie erwartet funktionieren. Durch E2E-Tests können Probleme erkannt werden, die in isolierten Unit- oder Integrationstests unter Umständen unentdeckt bleiben.
