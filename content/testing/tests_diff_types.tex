In der Softwareentwicklung gibt es mehrere verschiedene Arten von Tests, die in verschiedenen Phasen der Entwicklung eingesetzt werden.

\subsubsection{Unit-Tests}
Unit-Tests sind sehr einfach zu schreiben und testen die Anwendund nah an der Quelle. Mit ihnen werden die einzelnen Methoden und Funktionen der von der Software verwendeten Klassen, Komponenten oder Module getestet. Sie zielen darauf ab, dass jede Einheit des Codes isoliert und fehlerfrei arbeitet. In der Regel lassen sich Unit Tests auch sehr kostenkünstig automatisieren, so können sie von einem Continuous-Integration-Server sehr schnell durchgeführt werden.


\subsubsection{Integrationstests}
Das Ziel von Integrationstests ist es nicht wie die Unit-Tests die einzelnen Komponenten auf ihre Funktionalität zu prüfen, Integrationstests prüfen ob die einzelnen Module oder Komponenten auch ordnungsmeäß zusammenarbeiten. So kann die Interaktion mit der Datenbank oder das Zusammenspiel von anderen Services getestet werden. Im Falle dieser Diplomarbeit könnte man zum Beispiel testen ob Postmark auch mit der benutzten MongoDB Datenbank kompatibel ist.

\subsubsection{Funktionstests}
Diese Art von Tests sind darauf ausgerichtet, ähnlich wie die Unit-Tests die Funktionalität zu überprüfen, jedoch testen diese die Funktionalität der Anwendung alles ein Ganzes und nicht nur die einezlnen Komponenten.

\subsection{End-to-End-Tests}
E2E-Tests sind eine Art von Softwarestests, bei deinen ein großer Bereich der ganzen Anwendung getestet wird um sicherzustellen, dass die Anwendung auch in einer realen Umgebung ordnungsgemäß funktioniert. Der Sinn besteht darin, dass sichergestellt wird, dass alle Komponenten und Systeme der Applikation wie erwartet funktionieren. Durch E2E-Tests können Probleme erkannt werden, die in isolierten Unit- oder Integrationstests unter Umständen unentdeckt bleiben. Diese Art von Tests sind sehr nützlich, aber auch kostspielig und in automatisierter Form eher schwierig zu verwalten. Im Falle dieser Diplomarbeit wurde daher Cypress als Runner von den geschriebenen End-to-End Tests verwendet.


\cite{Verschiedene_Testarten}
https://www.atlassian.com/de/continuous-delivery/software-testing/types-of-software-testing