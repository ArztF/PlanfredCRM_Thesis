Grundlegend sind UI und UX dafür da, um Produkte wie Webseiten oder Apps, zu gestalten und konzipieren. UI bedeutet User Interface und ist für die Gestaltung eines Produktes zuständig. Ziel dabei ist ein ansprechendes und ästhetisches Design, welches einen User anspricht. UX bedeutet User Experience und sorgt dafür wie sich ein Produkt anfühlt, wie ein Benutzer:in damit kommuniziert und wie gut dies funktioniert. Es ist erkenntlich, dass beide Bereiche eng miteinander arbeiten und der Fokus dabei auf dem Erlebnis der Benutzer:innen liegt.

Die User Experience beschäftigt sich sehr stark mit den Erfahrungen und Gewohnheiten der Nutzer:innen. Dadurch soll eine Webseite so verständlich und logisch wie möglich gebaut werden. Eine schlechte User Experience wäre zum Beispiel, wenn man bei einer Änderung den Button zum Speichern bewusst suchen muss, Fehlermeldungen ohne tatsächlichen Lösungsweg angezeigt werden oder auch allgemein Elemente unlogisch und verwirrend angeordnet sind. Ein gutes UX-Design ist niht direkt bemerkbar, da in diesem Fall die Verwendung des Produktes optimal abläuft und somit keine Fragen oder Anregungen aufwirft.

Resultierend aus dem UX-Design wurde ein Grundgerüst gebaut, dieses muss dann visuell gestaltet werden. Dabei wird zum Beispiel zuerst verwendete Farben und Schriftarten festgelegt. Es ist wichtig, dass ein Design modern und ansprechend aussieht und das UX-Design ergänzt beziehungsweise unterstützt.

\cite{frontend_ui_ux}

\newpage