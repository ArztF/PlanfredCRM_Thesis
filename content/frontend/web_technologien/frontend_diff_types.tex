VueJS, React und Angular sind die drei weitverbreitetsten Web-Frameworks am Markt. Welches davon jedoch für ein Projekt am besten verwendet werden sollte, ist eine schwierige Frage und kann so nicht direkt beantwortet werden. Aus diesem Grund geht es in dem nächsten Abschnitt um die Unterschiede und jeweiligen Vor- und Nachteile.

Grundlegender Unterschied ist, dass React eine UI-Bibliothek ist, VueJS ein progressives Framework und Angular ein vollständiges Framework. Dennoch gibt es in der Entwicklung viele Ähnlichkeiten und einige gleiche Grundlagen.
Vorteil von Angular ist zum Beispiel die Vertrauenswürdigkeit. Angular wird von Google unterstützt und hat eine große Community, somit ist die Zukunft davon sehr sicher. Man kann sich darauf verlassen das Angular Projekte noch lange unterstützt und geupdatet werden. React wurde von Facebook entwickelt und ist damit auch ziemlich gut abgesichert. Bei VueJS ist dies jedoch nicht der Fall. Es steht keine große Firma oder überaus große Community dahinter und ist somit nicht ganz so vertrauenswürdig.

Dafür hat VueJS andere ausschlaggebende Vorteile. Beispielsweise ist die Lernkurve bei VueJS sehr klein und ist somit anfängerfreundlich. Dies ist bei React auch der Fall, Angular ist im Vergleich eher komplexer und schwieriger zum lernen.

Ebenso liefert VueJS beim Thema Projektgröße gut ab. Demnach sind VueJS Projekte vergleichsweise klein und bieten somit diverse Vorteile im Bezug auf SEO und Ladezeiten. Angular Projekte sind beispielsweise bemerkbar größer und React ordnet sich im Vergleich in der Mitte ein.

Zusammenfassend kann man sagen, dass alle drei Frameworks in verschiedenen Bereichen unterschiedlich gut abschneiden. Im Endeffekt ist jedoch keines grundlegend schlechter oder besser. Die Auswahl der richtigen Technologie richtet sich dabei viel mehr an andere Faktoren, wie Erfahrungen und Präferenzen des Entwicklerteams. Auch die Projektanforderungen können zu der richtigen Entscheidung beisteuern.
\cite{frontend_web_comparison}