Bei React handelt es sich nicht so wie bei VueJS um ein Framework, sondern um eine Sammlung an Bibliothek. Das sorgt dafür, das React im Vergleich sehr schlank aufgebaut ist. Bei React arbeitet man in normalen JavaScript Files, in welche man dann den HTML Code und die Logik einbindet. Erst später im Build-Prozess verwandelt die Bibliothek diesen Code in tatsächliche HTML Elemente. 

React ist schnell, das liegt an die Herangehensweise im Bezug auf den DOM. Normalerweise wird der DOM bei Änderungen immer komplett neu gerendert. Das dauert jedoch lange und ist vor allem nicht notwendig. Aus diesem Grund setzt die Bibliothek auf einen sogenannten virtuellen DOM. Das bedeutet, das im Browser-Zwischenspeicher der aktuelle DOM, also die Benutzeroberfläche gespeichert ist. Wenn eine Änderung auftritt vergleicht React den tatsächlichen DOM mit dem virtuellen DOM, entscheidet welche Bereiche tatsächlich geändert werden und gibt das an den DOM weiter. Das spart nicht nur Zeit sondern macht React auch Platform-unabhängig. Der virtuelle DOM kann auch in nativen IOS oder Android Code umgewandelt werden. Das ist mit dem Framework \textbf{React Native} möglich.

React arbeitet ebenfalls mit Komponenten.

\begin{lstlisting}[caption=React Komponente Grundgerüst]
    export default function App() {
      return (
        <div className="App">
          <h1>Hello World</h1>
        </div>
      );
    }
\end{lstlisting}

In diesem Beispiel sieht man einen simplen Aufbau einer React Komponente. Diese Komponente wird dann in die App gerendert. Um zu verstehen wie das funktioniert, muss zuerst die \textbf{index.html} Datei betrachtet werden. Diese Datei baut den Grundstein für die Darstellung im Browser.

\begin{lstlisting}
    <div id="root"></div>
\end{lstlisting}

Das ist der einzige tatsächliche HTML-Code der in einem ReactJS Projekt zu finden ist. Gerendert wird dieses Element in der Datei \textbf{index.tsx}.

\begin{lstlisting}
    const rootElement = document.getElementById("root");
\end{lstlisting}
\newpage
Damit wird das HTML-Element in eine JavaScript Variable gespeichert. Mit dieser Variable kann man nun Komponenten rendern, das sieht wie folgt aus:

\begin{lstlisting}
    render(<App />, rootElement);
\end{lstlisting}

Hierbei wird die App-Komponente in das root-Element gerendert.

\subsubsection{Installation}
Die Installation von React ist der erste Schritt, um mit der Entwicklung von React-Anwendungen zu beginnen. Eine der einfachsten Methoden, um ein neues React-Projekt zu erstellen, ist die Verwendung des Befehls \textbf{create-react-app}, der von Facebook bereitgestellt wird.

\begin{lstlisting}
    npx create-react-app meine-diplomarbeit
\end{lstlisting}

Dieser Befehl erstellt ein neues React-Projekt mit dem Namen \textbf{meine-diplomarbeit} und installiert alle erforderlichen Abhängigkeiten automatisch. Nach Abschluss des Installationsprozesses kann man in das Projektverzeichnis wechseln und mit der Entwicklung beginnen.

\cite{frontend_web_react_installation}

\subsubsection{Komponenten}
Komponenten sind die Bausteine von React-Anwendungen. Sie ermöglichen es Entwickler:innen, die Benutzeroberfläche in unabhängige, wiederverwendbare Teile zu unterteilen. Es gibt zwei Arten von Komponenten in React: Funktionskomponenten und Klassenkomponenten.

Eine Funktionskomponente ist eine JavaScript-Funktion, die ein React-Element zurückgibt:

\begin{lstlisting}[caption=Funktionskomponente]
    function MeinKomponent() {
      return <div>Hallo Welt!</div>;
    }
\end{lstlisting}
\newpage
Eine Klassenkomponente ist eine JavaScript-Klasse, die von \textbf{React.Component} erbt und eine \textbf{render()}-Methode enthält:

\begin{lstlisting}[caption=Klassenkomponente]
    class MeinKomponent extends React.Component {
      render() {
        return <div>Hallo Welt!</div>;
      }
    }
\end{lstlisting}

\cite{frontend_web_react_components}

\subsubsection{Property Binding}
Props (Eigenschaften) sind ein Mechanismus in React, um Daten von einer übergeordneten Komponente an eine untergeordnete Komponente zu übergeben. Props sind unveränderlich und werden in Funktionskomponenten als Parameter und in Klassenkomponenten über this.props zugänglich gemacht.

\begin{lstlisting}[caption=Property Binding Beispiel]
    class ElternKomponente extends React.Component {
      render() {
        return <KindKomponent name="Max" />;
      }
    }
    class KindKomponente extends React.Component {
      render() {
        return <div>Hallo {this.props.name}!</div>;
      }
    }
\end{lstlisting}

In diesem Beispiel wird der Wert \textbf{Max} als Prop namens \textbf{name} von der Elternkomponente zur Kindkomponente übergeben und dort verwendet.

\cite{frontend_web_react_components}

\subsubsection{Event Binding}
Event Binding ermöglicht es, Ereignisbehandlungsfunktionen an bestimmte Benutzerinteraktionen wie Klicks, Änderungen, Tastendrücke usw. zu binden. In React werden Ereignisse ähnlich wie in reinem JavaScript behandelt, jedoch mit einigen Unterschieden.
\newpage
\begin{lstlisting}[caption=Event Binding Beispiel]
    class MeinKomponent extends React.Component {
      handleClick() {
        console.log('Button wurde geklickt!');
      }
    
      render() {
        return (
          <button onClick={this.handleClick}>Klick mich</button>
        );
      }
    }
\end{lstlisting}

In diesem Beispiel wird die Methode \textbf{handleClick} an das \textbf{onClick}-Ereignis des Buttons gebunden. Wenn der Button geklickt wird, wird die Methode aufgerufen und eine Nachricht in der Konsole ausgegeben.

Diese Aspekte bilden die Grundlage für die Arbeit mit React. Mit einem fundierten Verständnis dieser Konzepte können Entwickler:innen robuste und skalierbare React-Anwendungen entwickeln. React bietet jedoch viele weitere Funktionen, die es ermöglichen, anspruchsvolle Benutzeroberflächen zu erstellen und effizient zu entwickeln.


\cite{frontend_web_react_events}