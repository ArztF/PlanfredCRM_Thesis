Zuerst ist abzuklären auf was sich der Begriff Frontend eigentlich bezieht.

“Das Frontend ist eine wichtige Komponente in der Welt des Webdesigns. Es bezeichnet die Benutzeroberfläche einer Website oder eines anderen digitalen Produkts, mit der Nutzerinnen interagieren können. Es bezieht sich auf alles, was die Userinnen sehen und berühren können, wenn sie das digitale Produkt verwenden.”
\newline
(vgl. \url{https://omr.com/de/daily/glossary/frontend}; Zugriff: 16.02.2024)

Das Frontend spielt bei einem Produkt eine entscheidende Rolle. Dabei ist es wichtig zu wissen, dass das Aussehen der Benutzeroberfläche mit dem steigenden Erwartungswert der Kunden:innen im Vergleich zu der Vergangenheit eine immer wichtigere Rolle spielt und somit auch ein fundamentales Grundprinzip in der Webentwicklung wurde. 

Es trägt dazu bei Kunden:innen zu gewinnen bzw. zu behalten. Das Aussehen eines Produktes bildet den Ersteindruck den ein Benutzer:in bei der Verwendung bekommt und muss demnach stimmig sein. Dementsprechend kann eine nicht ansprechende visuelle Darstellung zu einer Abneigung führen oder die Betroffenen abschrecken. 

Grundsätzlich kann man sagen, dass ein Frontend alle wichtigen Objekte, die auf einer Webseite oder generell Anwendung zu finden sind, beinhaltet und somit das visuelle Gesamtbild darstellt. Wichtig dabei ist auch die Kompatibilität auf diversen Browsern oder auch Geräten. Manche Browser unterstützen beispielsweise bestimmte neue Features noch nicht, dies muss in der Entwicklung berücksichtigt und die richtigen Alternativen gefunden werden. Verschiedene Geräte haben unterschiedliche Auflösungen und Seitenverhältnisse. Das Frontend soll hierbei auf jedem Produkt passend aussehen, dies nennt man Responsive Design und muss in der Entwicklung beachtet werden.

Für unser Kundenmanagement-Tool spielte das Frontend eine zentrale Rolle, da die Mitarbeiter:innen von Planfred tägliche mehrere Stunden das Produkt in Anspruch nehmen und somit eine schöne Benutzeroberfläche klar vom Vorteil ist.
\cite{frontend_grundlagen}
