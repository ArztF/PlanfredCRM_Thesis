\subsubsection{Single Page Application}
Unter einer Single Page Application oder auch kurz SPA versteht man eine Webseite die einmal vollständig geladen wird und dann bei Änderungen dynamisch aktualisiert wird. Das heißt es wird nicht mehr die ganze Seite neugeladen, sondern immer nur der Teil der sich wirklich verändert hat. Der größte Vorteil der dadurch auftritt ist eine bemerkbare Verbesserung der Ladezeiten. Wenn man beispielsweise auf eine Unterseite geht, wird nur der Inhalt der neuen Seite geladen. Inhalte, wie zum Beispiel Footer oder Navigation verändern sich bei einer solchen Aktion grundlegend nicht und werden dadurch bei einer SPA auch nicht neu geladen. Das spart viel Aufwand, da diese Bereiche nur einmal geladen werden müssen.
\cite{frontend_spa}

\subsubsection{Server Side Rendering}
Server Side Rendering oder auch kurz SSR ist eine Technik, bei der dem Browser vom Server einiges an Arbeit abgenommen wird um Zeit und Aufwand zu sparen. Dafür bereitet der Server HTML, CSS und Javascript so vor, dass der Browser diese nicht mehr vorbereiten muss, sondern nur noch rendern. Der Content wird also bereits renderfertig an den Browser gesendet.
Diese Technik hat einige Vorteile, wie zum Beispiel sehr schnelle Ladezeiten. Ein entscheidender Punkt ist ebenfalls eine signifikante Verbesserung im SEO-Bereich. Das liegt daran, dass die Suchmaschinen dabei bereits komplett fertigen HTML-Code durchsuchen können und das einen positiven Einfluss auf die Bewertung der Webseite hat. 
Ein Nachteil ist jedoch die erhöhte Last am Server. Das kann besonders bei sehr vielen Anfragen zu einem Problem werden.
\cite{frontend_ssr}

\subsubsection{Entscheidung}
Wir haben uns bei unserem Projekt bewusst für eine Single Page Application entschieden. Grund dafür ist, dass das Kundenmanagement Tool die perfekten Voraussetzungen dafür hat. Die verschiedenen Seiten unterscheiden sich kaum vom Design, es gibt nur ein grundlegendes Layout, welches sich nicht verändert. Lediglich der Content der angezeigt wird ändert sich, dabei gibt es jedoch auch keine großen Unterschiede und es ist alles einheitlich.

Ebenfalls wird unsere Webanwendung nur vom internen Support Team verwendet. Dadurch ist für uns eine Verbesserung der SEO-Performance vollkommen irrelevant.

\newpage