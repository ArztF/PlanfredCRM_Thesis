Im Fall dieser Projektarbeit, war bereits eine Datenbank gegeben, auf die das Projektteam zugreifen musste. Dennoch mussten Collections erstellt werden, auf die im Kapitel "Performance der Datenbank" näher eingegangen wird. Weiters werden in diesem Abschnitt die verwendeten Collections erläutert und ihre Schlüsselfelder, die für das Projekt notwendig sind, aufgelistet.
\newpage
\subsection{Users}
\begin{itemize}
    \item \_id
    \item email
    \item crm: Statistiken über Benutzer
        \begin{itemize}
            \item numberOfProjects: Anzahl der eigenen Projekte
        \end{itemize}
    \item payment: Zahlungsinformationen
        \begin{itemize}
            \item customerbillingplan
            \begin{itemize}
                \item plan: Ausgewähltes Abonnement
            \end{itemize}
        \end{itemize}
    \item enabled: Boolean, ob Benutzerin oder Benutzer gelöscht ist
    \item profile: Persönliche Benutzerdaten zum Beispiel Name, Adresse, ...
\end{itemize}

\subsection{Projects}
Folgende Fields waren für die Anzeige bzw. Abfragen notwendig:
\begin{itemize}
    \item \_id
    \item created
    \item participants: Array aller Benutzer:innen die in dem Projekt beteiligt sind
        \begin{itemize}
            \item \_id
            \item permission: Rolle welcher Benutzerin oder Benutzer in dem Projekt hat
            \item \_user\_id: ID der Benutzerin oder Benutzer
        \end{itemize}
    \item title
    \item usedStorage: Größe des Speicherplatzes, welches das Projekt benötigt
    \item lastRevisionUploadDocuments: Datum des letzten Dokumentuploads
    \item lastRevisionUploadPlans: Datum des letzen Planuploads
    \item trashed: Boolean, ob Projekt gelöscht ist
\end{itemize}