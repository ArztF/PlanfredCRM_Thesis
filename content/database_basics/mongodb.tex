\begin{figure}[h!]
    \centering
    \includegraphics[width=0.8\textwidth]{pics/mongodb.png}
    \caption{MongoDB Logo}
    \cite{database_mongodb_logo}
    \label{fig:enter-label}
\end{figure}

\textbf{Grundlagen}

MongoDB ist eine dokumentenbasierte Datenbank, die für eine einfache Anwendungsentwicklung und Skalierung ausgelegt ist. Dabei gibt es drei verschiedene Möglichkeiten MongoDB zu verwenden: 
\begin{itemize}
    \item \textbf{MongoDB Atlas}
        \newline
        Dieses ist ein Multi-Cloud Datenbankdienst, welcher das Deployen und Verwalten der Datenbank vereinfacht, aber gleichzeitig die Vielseitigkeit bietet, die zum Erstellen von leistungsstarken Anwendungen benötigt werden.
    \item \textbf{MongoDB Enterprise}
        \newline
        Dies ist ein abonnementbasierte und selbstverwaltete Version von MongoDB und umfasst mehr Funktionen:
        \begin{itemize}
            \item \textbf{LDAP-Authentifizierung}
                \newline
                LDAP bedeutet "lightweight directory access protocol" und hilft Benutzern beim Finden von Daten über Organisationen und Personen. LDAP-Authentifikation verfolgt dabei folgende Ziele: Daten im LDAP Vezeichnis zu speichern und Benutzer beim Zugriff auf dieses Verzeichnis zu authentifizieren. Dabei ist LDAP eines der wichtigsten Authentifizierungsprotokolle, welches für Verzeichnisdienste entwickelt wurde.
                \cite{ldap_auth}
            \item \textbf{Kerberos Authentifizierung}
                \newline
                Ist ein Sicherheitsprotkoll, welches mit dem KDC (Key Distribution Center) arbeitet, welches alle Clients, User und Dienste verwenden müssen, um zu authentifizieren. Außerdem wird beim Authentifizierungsprozess ein KDC-Ticket vergeben. Dieses Authentifizierungsverfahren ist die Standardmethode für das Betriebssystem Microsoft Windows.
                \cite{kerberos_auth}
            \item \textbf{Audit Events}
                \newline
                Audit-Events sind sicherheitsrelevante Ereignisse in einem System zum Beispiel ein Verstoß gegen Systemzugriffskontroll- oder Verantwortlichkeitssicherheitsrichtlinien und melden diese Verstöße an den System-Audit-Logger, welcher als Teil des Kernels ausgeführt wird. Es werden Name des Ereignisses, Erfolg oder Misserfolg und alle anderen ereignisspezifischen Informationen übermittelt.
                \cite{audit_events}
\end{itemize}
    \item \textbf{MongoDB Community}
        \newline
        Dies ist der Source Code von MongoDB, welcher kostenlos und selbstverwaltet zur Verfügung steht.
\end{itemize}
\cite{mongodb_basics}

\textbf{Aufbau der Datenbank}
\newline
Die Datenbank ist in folgende Teile unterteilt:

\begin{itemize}
    \item \textbf{Fields}
        \newline
        Dies sind die kleinste Einheit in der Datenstruktur und würden in einer relationalen Datenbank einer Entität ensprechen.
    \item \textbf{Documents}
        \newline
        Documents entsprechen in einer relationalen Datenbank einem Datensatz und werden mittels BSON (Binary JSON) kodiert. Im Vergleich zu JSON ist es allein durch die visuelle Inspektion der beiden Formate nicht möglich, einen Unterschied festzustellen, da BSON auf dem JSON-Format aufbaut.
        \newline
        Der entscheidende Unterschied zwischen BSON und JSON liegt darin, dass BSON zusätzliche Datenformate wie Datumswerte und Binärdaten unterstützt. Dies wird durch kodierte Typ- und Längeninformationen ermöglicht, was wiederum zu einer erheblichen Leistungssteigerung beim Durchlaufen des Datensatzes führt. Die Integration von BSON ermöglicht es auch, Arrays und andere Dokumente effizient in diese Datensätze einzufügen.
        \newline
        Der Vorteil von "Documents" sind:
        \begin{itemize}
            \item Dokumente entsprechen in einigen Programmiersprachen nativen Datentypen
            \item Die Einbindung von Arrays und Documents vermeidet aufwendige JOINS
            \item Durch das dynamische Schema kann man die Datenbankstruktur sehr vielseitig gestalten
        \end{itemize}
        \begin{figure}[h!]
            \centering
            \includegraphics[width=0.8\textwidth]{pics/document.png}
            \caption{Document in MongoDB}
            \cite{mongodb_document}
            \label{fig:enter-label}
        \end{figure}
        \cite{mongodb_json_vs_bson}
    \item \textbf{Collections}
        \newline
        Die "Documents" werden in sogenannten "Collections" gespeichert, welche vergleichbar mit Tabellen in relationalen Datenbanken sind. Collections besitzen eine sogenannte "dynamic schema" Eigenschaft, was bedeutet, dass "Documents" in einer "Collection" unterschiedliche "Fields" besitzen können. Wie man im unten gezeigten Beispiel sieht, können beide dieser Documents in der selben Collection gespeichert werden.
        \begin{lstlisting}
            {"username": "admin", "firstname": "John"}
            {"lastname": "Doe"}
        \end{lstlisting}
        Man könnte also ausschließlich mit einer einzigen Collection arbeiten und in diese alle Datensätze speichern, man würde allerdings ziemlich schnell an seine Grenzen stoßen.
    \item \textbf{Cluster}
        \newline
        In einem Cluster sind alle Server der Datenbank gespeichert in denen sich die Collections befinden. Diese Cluster können Replica Sets sein, also Kopien aller Daten, oder Sharded-Cluster sein. Zu diesem Cluster wird im Laufe dieses Kapitels noch genauer eingegangen.
        \cite{mongodb_collections}
\end{itemize}

\textbf{Skalierung}
\newline
Die die Anzahl der Daten die gespeichert werden im Laufe der Zeit immer weiter anwächst, ist es wichtig seine Datenbank einfach skalieren zu können. 
\newline
\textbf{Horizontale Skalierung}
\newline
MongoDB wurde auf die horizontale Skalierung ausgelegt. Dazu müssen neue Maschinen (Server) dazugekauft werden, auf denen anschließend die Daten verteilt werden. Durch das dokumentenbasierte Datenmodell ist es wesentlich einfacher die Daten auf verschiedene Maschinen zu verteilen, dabei achtet MongoDB selbst darauf, dass alle Server annähernd gleich ausgelastet sind. Dazu werden Daten auch von einem Server auf einen anderen Server verschoben. Dazu verwendet MongoDB Shard-Schlüssel, welcher als Field in den Documents gespeichert wird und dient dazu die Documents in den Collections zu verteilen.
\cite{mongodb_collections}

Wie in der untenstehenden Grafik zu sehen, wird hier von "Shards" gesprochen. Das sogenannte Sharding stammt aus der Welt der traditionellen Datenbanken und beschreibt das zerteilen einer großen Datenbank in mehrere kleine. Die neu entstandenen kleinen Datenbanken nennt man Shards. Der Vorteil dieser Methode sind schnellere Schreib- und Lesezugriffe, da alle Shards in einem Cluster diese Befehle ausführen, wodurch ein hoher Grad der Parallelität erreicht wird.
\begin{figure}[h!]
    \centering
    \includegraphics[width=0.6\textwidth]{pics/vertical_scaling_mongodb.png}
    \caption{Horizontale Skalierung MongoDB}
    \cite{vertical_scaling_mongodb}
    \label{fig:enter-label}
\end{figure}
\newline
MongoDB unterstützt zwei unterschiedliche Sharding-Methoden:
\begin{itemize}
    \item \textbf{Hashed Sharding}
        \newline
        Bei dieser Methode wird ein Hashwert aus dem Shard-Schlüssel gebildet und anschließend wird jedem Block ein Bereich zugewiesen, der auf den gehashten Wert basiert.
        \begin{figure}[h!]
            \centering
            \includegraphics[width=0.5\textwidth]{pics/hashed_sharding.png}
            \caption{Hashed Sharding MongoDB}
            \cite{hashed_sharding_image}
            \label{fig:enter-label}
        \end{figure}
    \item \textbf{Ranged Sharding}
        \newline
        Bei dieser Methode werden die Daten aufgrund der Shard Schlüsselwerten in Bereiche unterteilt. Anschließend wird auf Basis der Shard-Schlüsselwerte  jedem Chunk ein entsprechender Bereich zugewiesen.
        \begin{figure}[h!]
            \centering
            \includegraphics[width=0.5\textwidth]{pics/ranged_sharding.png}
            \caption{Ranged Sharding MongoDB}
            \cite{range_sharding_image}
            \label{fig:enter-label}
        \end{figure}
\end{itemize}
\cite{mongodb_sharding}

\textbf{Vertikale Skalierung}
\newline
Es gibt allerdings auch die Möglichkeit eine MongoDB Datenbank vertikal zu Skalieren. Dabei wird die Kapazität eines einzigen Servers gesteigert, indem man eine stärkere CPU, mehr RAM oder mehr Speicherplatz einbaut. In diesem Fall ist man allerdings an den technologischen Fortschritt angewiesen und es ist nicht unendlich möglich diese Skalierungsvariante zu verwenden. Des weiteren ist in den meisten Fällen eine horizontale Skalierung wesentlich kosteneffizienter als eine vertikale.