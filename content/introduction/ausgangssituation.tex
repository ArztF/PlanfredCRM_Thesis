Vor allem in der heutigen Zeit der Digitalisierung ist es für jede Firma essentiell, eine strukturierte Einsicht auf Kundendaten zu besitzen. Auch das Supportteam von Planfred benötigt ein neues, effizienteres und übersichtlicheres System, welches Kunden bei Problemen oder Wünschen zur Seite stehen sollte. Grund für die Erneuerung ist, dass bei dem vorigen Kunden Monitoring Toll zahlreiche Probleme beziehungsweise Einschränkungen aufgetreten sind.
\newline
Grundlegend verfügte die Firma Planfred über eine veraltete AngularJS 1.0 Anwendung. Auf Grund dessen gab es Komplikationen mit der Unübersichtlichkeit der Benutzerfläche und der mangelhaften UX, was die Arbeit der Mitarbeiter stark verlangsamte. Darüber hinaus benötigen Anfragen eine Ewigkeit, was an großen Performanceproblemen der Anwendung gelegen ist. Der Prozess, einen Kunden zu helfen, der Hilfe benötigt, wurde dadurch unnötigerweise verzögert und führte weitergehend zu Unzufriedenheit bei den Kunden.
\newline
Ein weiteres ausschlaggebendes Problem lag darin, dass bei der veralteten Version nur Read-Operationen ausführbar waren. Dementsprechend konnte das Supportteam bei Support-Anfragen keine Änderungen an den Kundendaten vornehmen. Diese essentielle Funktion des Änderns und Bearbeiten von Abonnements und einzelner Kundendaten, sollte in dieser Diplomarbeit umgesetzt werden. Des Weiteren sollten die Mitarbeiter:innen die Möglichkeit haben, Projekte zu löschen beziehungsweise wiederherzustellen und den verfügbaren Speicherplatz zu ändern.