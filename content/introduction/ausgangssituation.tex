Vor allem in der heutigen Zeit der Digitalisierung ist es f\"ur jede Firma essentiell, eine strukturierte Einsicht auf Kundendaten zu besitzen. So auch für Planfred, einem Unternehmen, welches ein Software Tool f\"r die Baubranche anbietet. Dieses Produkt hilft für die Dokumentenverwaltung von Bauprojekten. Auch das Supportteam von Planfred benötigt ein neues, effizienteres und \"ubersichtlicheres System, welches Kunden:innen bei Problemen oder W\"unschen zur Seite stehen sollte. Grund für die Erneuerung ist, dass bei dem vorigen Kunden-Monitoring-Tool zahlreiche Probleme beziehungsweise Einschränkungen aufgetreten sind.
\newline
Grundlegend verf\"ugte die Firma Planfred \"uber eine veraltete AngularJS 1.0 Anwendung. Auf Grund dessen gab es Komplikationen mit der Un\"ubersichtlichkeit der Benutzerfläche und der mangelhaften UX, was die Arbeit der Mitarbeiter:innen stark verlangsamte. Dar\"uber hinaus benötigen Anfragen eine Ewigkeit, was an großen Performanceproblemen der Anwendung gelegen ist. Der Prozess, einem Kunden:in zu helfen, der oder die Hilfe benötigt, wurde dadurch unnötigerweise verzögert und f\"uhrte weitergehend zu Unzufriedenheit bei den Kunden:innen.
\newline
Ein weiteres ausschlaggebendes Problem lag darin, dass bei der veralteten Version nur Read-Operationen ausf\"uhrbar waren. Dementsprechend konnte das Supportteam bei Support-Anfragen keine Änderungen an den Kundendaten vornehmen. Diese essentielle Funktion des Änderns und Bearbeiten von Abonnements und einzelner Kundendaten, sollte in dieser Diplomarbeit umgesetzt werden. Des Weiteren sollten die Mitarbeiter:innen die Möglichkeit haben, Projekte zu löschen beziehungsweise wiederherzustellen und den verf\"ugbaren Speicherplatz zu ändern.
\newpage