Als Kundenmanagement Tool kann unsere Diplomarbeit vom Planfred Supportteam verwendet werden und den Prozess der Kundenunterstützung erleichtern beziehungsweise unterstützen.
Im folgenden Abschnitt sind die Anwendungsfälle dabei nochmal ausführlich erklärt:

\begin{itemize}
    \item \textbf{Suche Domain}
        \begin{itemize}
            \item Ein potenzieller neuer Kunde:in erstellt ein erstes Projekt und sucht nach relevanten Informationen über eine Domain, einschließlich der Beteiligten, Berechtigungen, letzter Aktivitäten und Kontoinformationen.
            \item  Ein Kunde:in möchte Projekte, die von Mitarbeiter:innen angelegt wurden, an eine unpersonifizierte E-Mailadresse übertragen und benötigt eine Suche nach Projekteigentümer:innen und deren Projekten.
        \end{itemize}
    \item \textbf{Suche E-Mailadresse}
        \newline
        Ein Mitarbeiter:in eines Kunden:in sucht nach Projekten, an denen er beteiligt ist, und deren Berechtigungen.
    \item \textbf{Live-Suche nach E-Mailadressen und Projekttiteln}
        \newline
        Suchfunktion, die nach E-Mailadresses oder Projekttitel filtert und Live-Rückmeldungen während der Eingabe liefert.
    \item \textbf{Suche nach Benutzer:innen mit ausgewähltem Paket}
        \newline
        Suche nach Benutzer:innen basierend auf ihrem Paket, mit Statistiken zu Projekteigentümer:innen, Speicherplatz und Projektdetails.
    \item \textbf{Selbst registriert ohne Projekt und Beteiligungen}
        \newline
        Liste von Benutzer:innen, die sich registriert haben, aber noch keine Projekte angelegt haben, für den Versand von Informationen.
    \item \textbf{Speicherlimit überschritten}
        \newline
        Liste von Projekteigentümer:innen, deren Speicherlimit überschritten wurde, mit Details zu ihren Projekten und dem verbrauchten Speicherplatz.
    \item \textbf{Allgemeine Kennzahlen auf Startseite}
        \newline
        Anzeige von allgemeinen Kennzahlen auf der Startseite, ergänzt um Informationen über Eigentümer:innen.
    \item \textbf{Spezialpaket mit anderem Speicherplatz}
        \newline
        Suche nach Kunden:innen mit speziellen Paketen und Dropdown-Auswahl für Speicherplatz.
    \item \textbf{Änderungen müssen dokumentiert werden}
        \newline
        Automatische Generierung von E-Mails zur Dokumentation von Änderungen.
    \item \textbf{Warum sieht ein Benutzer das Projekt nicht mehr}
        \newline
        Suche nach Gründen, warum ein Benutzer:innen ein Projekt nicht mehr sehen kann, und Lösungsmöglichkeiten.
    \item \textbf{Projekt löschen und wiederherstellen}
        \newline
        Sicherstellung, dass versehentlich gelöschte Projekte wiederhergestellt werden können und Dokumentation von Löschvorgängen.
    \item \textbf{Payment Plan auf "None" umstellen}
        \newline
        Möglichkeit, den Zahlungsplan auf "None" umzustellen, wenn ein Benutzer:in kein Eigentümer:in mehr ist.
    \item \textbf{Auch nach gelöschten Projekt-Beteiligten suchen}
        \newline
        Suche nach gelöschten Projektbeteiligten und deren Anzeige.
    \item \textbf{Suche nach gelöschten Benutzerkonten}
        \newline
        Möglichkeit, nach deaktivierten Benutzerkonten zu suchen.
\end{itemize}
\newpage