\begin{spacing}{1}
    \chapter*{Abstract}
\end{spacing}
\begin{wrapfigure}{r}{0.3\textwidth}
    \begin{center}
      \includegraphics[width=0.2\textwidth]{pics/question_mark.png}
    \end{center}
\end{wrapfigure}
PlanfredCRM is a web app created by Felix Arzt, Timo Mittermayr, and Nico Obermair during their diploma project with Planfred GmbH. Planfred itself is software designed for managing documents, plans, and tasks in the construction industry. Because of the subscription model, it's important for support and sales staff to keep an eye on Planfred users, especially regarding their storage usage and project numbers.

PlanfredCRM helps with this. It lets staff see and edit user data from the main database. With PlanfredCRM, staff can adjust storage packages, increase or decrease storage space, and recover deleted projects or users. They can also reach out to users if they're close to exceeding storage limits, offering new options to keep things running smoothly.

The frontend of PlanfredCRM uses VueJS, while the backend, which connects to the MongoDB production database, is powered by a NodeJS server.
\newpage
\begin{spacing}{1}
    \chapter*{Zusammenfassung}
\end{spacing}
\begin{wrapfigure}{r}{0.3\textwidth}
    \begin{center}
      \includegraphics[width=0.2\textwidth]{pics/question_mark.png}
    \end{center}
\end{wrapfigure}
PlanfredCRM ist eine Web-Applikation, die von Felix Arzt, Timo Mittermayr und Nico Obermair im Rahmen der Diplomarbeit mit der Firma Planfred GmbH entstanden ist. Planfred ist ein eigenes Softwareprodukt, welches ein Dokumenten-, Pläne- und Taskverwaltung für die Baubranche darstellt. Für die Mitarbeiter:innen im Support und Vertrieb ist es aufgrund des vorhandenen Abomodells wichtig, die Benutzer: innen von Planfred zu überwachen, vor allem in Punkten wie verbrauchter Speicherplatz und Anzahl der eigenen Projekte. An dieser Stelle kommt PlanfredCRM ins Spiel, dieses Tool hilft den Mitarbeiter: innen die Daten der Benutzer: innen aus der Produktionsdatenbank einzusehen bzw. zu verändern. Mit dieser Applikation ist es möglich, gebuchte Speicherpakete auf Kundenwunsch zu verändern, den jeweiligen Speicherplatz zu erhöhen bzw. zu verringern. Gelöschte Projekte oder Benutzer: innen können wiederhergestellt werden. Des Weiteren können die Mitarbeiter: innen von sich aus Kunden im Fall einer Speicherplatzüberschreitung mit einem neuen Angebot kontaktieren, um die Arbeit mit Planfred weiterhin bestmöglich zu garantieren.

Die Visualisierung wurde im Frontend mithilfe von VueJS umgesetzt. Im Backend läuft ein NodeJS-Server, welche gleichzeitig auch die Schnittstelle zur MongoDB Produktionsdatenbank ist.

