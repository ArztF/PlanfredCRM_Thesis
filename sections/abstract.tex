\begin{spacing}{1}
    \chapter*{Abstract}
\end{spacing}
\begin{wrapfigure}{r}{0.3\textwidth}
    \begin{center}
      \includegraphics[width=0.2\textwidth]{pics/question_mark.png}
    \end{center}
\end{wrapfigure}
PlanfredCRM is a web application that was developed by Felix Arzt, Timo Mittermayr, and Nico Obermair as part of their thesis in collaboration with Planfred GmbH. Planfred is a software product that provides document, plan, and task management for the construction industry. Planfred operates with various subscription models, each differing in functionality. Due to the constantly changing requirements in construction projects, these packages need to be modified or adjusted, leading to ongoing communication between customers and the support team. This is where PlanfredCRM comes into play. This tool assists employees in viewing or modifying user data from the production database. With this application, it's possible to modify booked storage packages according to customer preferences, increase or decrease the respective storage space. Additionally, deleted projects or users can be restored. Furthermore, employees can proactively contact customers in case of storage space exceeded to provide them with a new offer, ensuring the optimal continuation of work with Planfred.

The visualization was implemented in the frontend using VueJS. The backend is based on a NodeJS server, which also serves as the interface to the MongoDB production database.
\newpage
\begin{spacing}{1}
    \chapter*{Zusammenfassung}
\end{spacing}
\begin{wrapfigure}{r}{0.3\textwidth}
    \begin{center}
      \includegraphics[width=0.2\textwidth]{pics/question_mark.png}
    \end{center}
\end{wrapfigure}
PlanfredCRM ist eine Web-Applikation, die von Felix Arzt, Timo Mittermayr und Nico Obermair im Rahmen der Diplomarbeit mit der Firma Planfred GmbH entstanden ist. Planfred ist ein eigenes Softwareprodukt, welches eine Dokumenten-, Pläne- und Taskverwaltung für die Baubranche darstellt. Planfred arbeitet dabei mit verschiedenen Abo Modellen, welche sich im Funktionsumfang unterscheiden. Da diese Pakete aufgrund der ständig wechselnden Anforderungen im Bauprojekten geändert oder angepasst werden müssen, entsteht eine laufende Kommunikation zwischen Kunden und Supportteam. An dieser Stelle kommt PlanfredCRM ins Spiel, dieses Tool hilft den Mitarbeiter: innen die Daten der Benutzer: innen aus der Produktionsdatenbank einzusehen bzw. zu verändern. Mit dieser Applikation ist es möglich, gebuchte Speicherpakete auf Kundenwunsch zu verändern, den jeweiligen Speicherplatz zu erhöhen bzw. zu verringern. Ebenfalls können gelöschte Projekte oder Benutzer: innen wiederhergestellt werden. Des Weiteren können die Mitarbeiter: innen von sich aus Kunden im Fall einer Speicherplatzüberschreitung mit einem neuen Angebot kontaktieren, um die Arbeit mit Planfred weiterhin bestmöglich zu garantieren.

Die Visualisierung wurde im Frontend mithilfe von VueJS umgesetzt. Das Backend basiert dabei auf einem NodeJS-Server, welcher gleichzeitig auch die Schnittstelle zur MongoDB Produktionsdatenbank ist.

