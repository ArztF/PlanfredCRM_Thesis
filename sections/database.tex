In diesem Kapitel werden zunächst die Grundlagen und die Definition der Datenbank erläutert. In den weiteren Kapiteln wird ein Vergleich zwischen relationalen und objektorientierten Datenbanken aufgestellt und wie es zu der Entscheidung für MongoDB gekommen ist. Des weiteren wird es um die Performance der Abfragen gehen und auf die Lösung mittels Cron-Jobs und die Implementierung eingegangen.

\section{Grundlagen}
\setauthor{Felix Arzt}
"Eine Datenbank ist eine organisierte Sammlung von strukturierten Informationen oder Daten, die typischerweise elektronisch in einem Computersystem gespeichert sind. Eine Datenbank wird normalerweise von einem Datenbankverwaltungssystem (DBMS) gesteuert."
\newline
(\url{https://www.oracle.com/de/database/what-is-database/}; Zugriff: 20.02.2024)
\newline
Dabei gibt es verschiedene Arten von Datenbanktypen. Im Zuge dieser Diplomarbeit wird lediglich die relationale und die objektorentierte Datenbank von Relevanz. Im folgenden Kapitel wird nun näher auf den Unterschied zwischen den beiden Typen eingegangen wird auf die Gründe, warum es zur Entscheidung für MongoDB gekommen ist.
\cite{database_basics}

\section{Vergleich}
\setauthor{Felix Arzt}
\textbf{Relationale Datenbank}
\newline
Eine relationale Datenbank basiert auf einem relationalen Modell und stellt die Daten in einer Tabelle dar. Dabei ist jede Zeile in dieser Tabelle genau ein Datensatz, welcher mit einer eindeutigen ID (Schlüssel) versehen ist. Die Spalten der Tabelle stellen die Attribute dar, welche gespeichert werden können. Bei einer relationalen Datenbank sind die logischen Datenstrukturen, sprich die Datentabellen, Ansichten und Indizes, von den physischen Datenstrukturen getrennt. Dies ermöglicht es den Datenbankadministrator bzw. die Datenbankadministratorin physische Datenstrukturen zu verändern ohne dabei die logische Datenstruktur zu beeinträchtigen. 
\newline
Der wohl größte Vorteil einer relationalen Datenbank ist die \textbf{ACID-Eigenschaft}, dabei steht ACID für: \textbf{A}tomicity, \textbf{C}onsistency, \textbf{I}solation und \textbf{D}urability.

\begin{itemize}
    \item \textbf{Atomicity}
        \newline
        Atomicity beschreibt die sogenannten Regeln in einer Datenbank, um einen Datensatz einzufügen. Wird eine der Regeln nicht erfüllt, wird der Datensatz nicht eingefügt. Die Atomarität beschreibt also den Schlüssel, welcher sicherstellt, dass alle Daten die sich in der Datenbank befinden den Regeln entsprechen.
    \item \textbf{Consistency}
        \newline
       Consistency ist dafür da, dass in allen Instanzen (Kopien der Datenbank) zu jeder Zeit, immer die gleichen Daten zur Verfügung stehen.
    \item \textbf{Isolation}
        \newline
        Isolation beschreibt den Vorgang, dass jede Transaktion verborgen von den anderen abläuft und erst nach beenden der Transaktion in der Datenbank "veröffentlicht" wird. Dadurch wird verhindert, dass sich die Transaktionen gegenseitig beeinflussen bzw. beeinträchtigen und trotzdem parallel ablaufen können.
    \item \textbf{Durability}
        \newline
        Durability ist die Versicherung, dass alle Daten dauerhaft gespeichert werden, sobald die Transaktion abgeschlossen ist.
\end{itemize}
Zusammenfassend lässt sich sagen, dass eine relationale Datenbank besonders geeignet ist, wenn eine Vielzahl von Daten in einer sicheren, regelbasierten (Atomicity) und konsistenten (Consistency) Weise gespeichert werden sollen, insbesondere wenn diese Daten untereinander in Beziehung stehen. In diesem speziellen Fall der vorliegenden Diplomarbeit war dies durchaus eine Anforderung, da zahlreiche Daten gespeichert werden mussten und eine klare Beziehung zwischen den einzelnen Tabellen vorhanden war.
\newline
Trotz dieser Anforderungen entschied sich das Projektteam aufgrund des umfangreichen Know-hows in der Kollaborationsfirma für den Einsatz von objektorientierten Datenbanken. Diese Entscheidung führte zur Auswahl von MongoDB als Datenbanktyp, da die benötigten Daten bereits in einer MongoDB-Datenbank gespeichert waren. Ein Wechsel zu einer relationalen Datenbank wäre aufgrund des damit verbundenen Aufwands nicht gerechtfertigt gewesen.
\newline
Rückblickend ist anzumerken, dass sich bei der Verwendung einer relationalen Datenbank viele Abfragen wesentlich einfacher gestaltet hätten. Einige Abfragen erforderten in MongoDB die Durchführung eines JOINs über zwei Collections, was sich als durchaus anspruchsvoll erwies.
\cite{database_relational}
\newline
\begin{figure}[h!]
  \centering
  \includegraphics[width=0.8\textwidth]{pics/database-types.jpg}
  \caption{Relational vs. Objektorientiert}
  \cite{database_types}
\end{figure}

\section{MongoDB}
\setauthor{Felix Arzt}
\begin{figure}[h!]
    \centering
    \includegraphics[width=0.8\textwidth]{pics/mongodb.png}
    \caption{MongoDB Logo}
    \cite{database_mongodb_logo}
    \label{fig:enter-label}
\end{figure}
MongoDB ist eine dokumentenbasierte Datenbank, die für eine einfache Anwendungsentwicklung und Skalierung ausgelegt ist. Dabei gibt es drei verschiedene Möglichkeiten MongoDB zu verwenden: 
\begin{itemize}
    \item \textbf{MongoDB Atlas}
        \newline
        Dieses ist ein Multi-Cloud Datenbankdienst, welcher das Deployen und Verwalten der Datenbank vereinfacht, aber gleichzeitig die Vielseitigkeit bietet, die zum Erstellen von leistungsstarken Anwendungen benötigt werden.
    \item \textbf{MongoDB Enterprise}
        \newline
        Dies ist ein abonnementbasierte und selbstverwaltete Version von MongoDB und umfasst mehr Funktionen:
        \begin{itemize}
            \item \textbf{LDAP-Authentifizierung}
                \newline
                LDAP bedeutet "lightweight directory access protocol" und hilft Benutzern beim Finden von Daten über Organisationen und Personen. LDAP-Authentifikation verfolgt dabei folgende Ziele: Daten im LDAP Vezeichnis zu speichern und Benutzer beim Zugriff auf dieses Verzeichnis zu authentifizieren. Dabei ist LDAP eines der wichtigsten Authentifizierungsprotokolle, welches für Verzeichnisdienste entwickelt wurde.
                \cite{ldap_auth}
            \item \textbf{Kerberos Authentifizierung}
                \newline
                Ist ein Sicherheitsprotkoll, welches mit dem KDC (Key Distribution Center) arbeitet, welches alle Clients, User und Dienste verwenden müssen, um zu authentifizieren. Außerdem wird beim Authentifizierungsprozess ein KDC-Ticket vergeben. Dieses Authentifizierungsverfahren ist die Standardmethode für das Betriebssystem Microsoft Windows.
                \cite{kerberos_auth}
            \item \textbf{Audit Events}
                \newline
                Audit-Events sind sicherheitsrelevante Ereignisse in einem System zum Beispiel ein Verstoß gegen Systemzugriffskontroll- oder Verantwortlichkeitssicherheitsrichtlinien und melden diese Verstöße an den System-Audit-Logger, welcher als Teil des Kernels ausgeführt wird. Es werden Name des Ereignisses, Erfolg oder Misserfolg und alle anderen ereignisspezifischen Informationen übermittelt.
                \cite{audit_events}
\end{itemize}
    \item \textbf{MongoDB Community}
        \newline
        Dies ist der Source Code von MongoDB, welcher kostenlos und selbstverwaltet ist.
\end{itemize}
\cite{mongodb_basics}

Dadurch das MongoDB ein dokumentenbasierte Datenbank ist, sind Datensätze wie man sie in SQL kennt, sogenannte Documents, welche aus mehreren Fields bestehen. Diese Fields sind die Entitäten des Datensatzes und werden als BSON (Binary JSON) dargestellt. Der Unterscheid zwischen BSON und JSON ist alleine durch betrachten der beiden Formate nicht möglich, da BSON auf JSON aufbaut. Der Unterschied liegt darin, dass in BSON Datenformate, wie Dates und Binärdaten hinzugefügt wurden und durch kodierte Typ- und Längeninformationen es wesentlich performanter ist den Datensatz zu durchlaufen. Durch BSON ist es auf möglich Arrays und andere Documents in diesen Datensatz einzufügen.
\cite{mongodb_json_vs_bson}

\begin{figure}[h!]
    \centering
    \includegraphics[width=0.8\textwidth]{pics/document.png}
    \caption{Document in MongoDB}
    \cite{mongodb_document}
    \label{fig:enter-label}
\end{figure}

Der Vorteil von Documents sind:
\begin{itemize}
    \item Dokumente entsprechen in einigen Programmiersprachen nativen Datentypen
    \item Die Einbindung von Arrays und Documents vermeidet aufwendige JOINS
    \item Durch das dynamische Schema kann man die Datenbankstruktur sehr vielseitig gestalten
\end{itemize}

\section{Datenbankstruktur}
\setauthor{Felix Arzt}
In dem Fall dieser Projektarbeit, war bereits eine Datenbank gegeben, auf die das Projektteam zugreifen musste. Dennoch mussten Collections erstellt werden, die Begründung dafür, wird im Kapitel "Performance der Datenbank" erläutert. In diesem Abschnitt werden die verwendeten Collections erläutert und ihre Schlüsselfelder, die für das Projekt notwendig war, aufgelistet.

\subsubsection{Users}
\begin{itemize}
    \item \_id
    \item email
    \item crm: Statistiken über Benutzer
        \begin{itemize}
            \item numberOfProjects: Anzahl der eigenen Projekte
        \end{itemize}
    \item payment: Zahlungsinformationen
        \begin{itemize}
            \item customerbillingplan
            \begin{itemize}
                \item plan: Ausgewähltes Abonnement
            \end{itemize}
        \end{itemize}
    \item enabled: Boolean, ob Benutzer:in gelöscht ist
    \item profile: Persönliche Benutzerdaten zum Beispiel Name, Adresse, ...
\end{itemize}

\subsubsection{Projects}
\begin{itemize}
    \item \_id
    \item created
    \item participants: Array aller Benutzer:innen die in dem Projekt beteiligt sind
        \begin{itemize}
            \item \_id
            \item permission: Rolle welcher Benutzer:in in dem Projekt hat
            \item \_user\_id: ID der Benutzerin / des Benutzers
        \end{itemize}
    \item title
    \item usedStorage: Größe des Speicherplatzes, welches das Projekt benötigt
    \item lastRevisionUploadDocuments: Datum des letzten Dokumentuploads
    \item lastRevisionUploadPlans: Datum des letzen Planuploads
    \item trashed: Boolean, ob Projekt gelöscht ist
\end{itemize}
Dies sind nicht alle Fields in den Collections, jedoch die wesentlichen, welche für die Abfragen bzw. die Anzeige notwendig waren.

\section{MongoDB Indexes}
\setauthor{Nico Obermair}
Ein MongoDB Search Index, dient um die Daten in einer logischen Reihenfolge zu katalogiseren, damit diese schneller gefunden werden können.

Die Implementierung eines Search Indexes ist nicht mehr als ein Klick in MongoDB Atlas. Die Anwender:innen gehen zu einem Cluster deren Wahl und klicken dort auf den Button "Suchindex erstellen".

Sobald dieser Schritt erledigt ist, kann in der MongoDB Abfrage der 

\begin{lstlisting}
    $search 
\end{lstlisting}

Operator verwendet werden.

\begin{lstlisting}
    {
      $search: {
        index: 'searchUsers',
        text: {
          query: queryParams.searchParam,
          path: {
            wildcard: '*'
          }
        }
      }
    }
\end{lstlisting}

Das Property "index" gibt an auf welchen Index referenziert wird, da auch mehrere Indizes vorhanden sein können. 
Der Text nach welchem gesucht wird ist definiert unter "text.query". In dieser Diplomarbeit wird der Text nach welchem gesucht wird mittles Query Parametern übergeben. 

Mit dem Path Attribut wird definiert, über welche Felder der Index sucht. In diesem Fall werden alle Felder durchsucht, da es sich um eine Fulltext Search handelt.

MongoDB verwendet für den SearchIndex Apache Lucene, somit müssen die Entwickler:innen, diese Engine nicht mehr manuell einbinden. 

Apache Lucene stellt mehrere verschiedene Analyzer zur Verfügung, welche nach verschiedenen Kriterien filtern.

\begin{itemize}
    \item \textbf{Standard (Default)}
        \newline
        Der Standard Analyzer bietet eine grammatikbasierte Tokenisierung welche E-Mail Adressen und alphanumerische Zeichen erkennt
    \item \textbf{English}
        \newline
        Der English Analyzer berücksichtigt die englische Sprache, bei diesem werden Wörter wie "and", "the", "is" usw. entfernt.
    \item \textbf{Simple}
        \newline
        Der Simple Analyzer filtert den Text auf jedes Zeichen welches kein Buchstabe ist. Leerzeichen, Satzzeichen und Ziffern werden entfernt.
    \item \textbf{Whitespace}
        \newline
        Der Whitespace Analyzer filtert den Text nur auf Leerzeichen. Er belässt alle Begriffe in ihrer originalen Schreibweise und behält die Zeichensetzung bei.
    \item \textbf{Keyword}
        \newline
        Dieser Analyzer sucht nacht exakten Treffern. Groß-/Kleinschreibung und Zeichensetzung bleiben hier völlig erhalten.
    \item \textbf{French}
        \newline
        Diese Analyzer ist ähnlich wie der English Analyzer, mit dem Unterschied das er nach populären Wörtern in der französischen Sprache sucht wie zum Beispiel "le", "au", "mon".
\end{itemize}





In dieser Diplomarbeit wurde der MongoDB Search Index verwendet um alle Felder einer Anwender:in auf Schlüsselwörter zu durchsuchen.

\begin{figure}[h!]
    \centering
    \includegraphics[width=1\linewidth]{pics/mongodb-search-indizes.png}
    \caption{MongoDB Search Index}
    \label{fig:enter-label}
\end{figure}


\cite{Search_Indexes}
\cite{Atlas_Indexes}




\section{Performance der Datenbank}
\setauthor{Felix Arzt}
Die Performance der Datenabfragen ist von essenzieller Bedeutung, insbesondere im Bereich des Supports, da Mitarbeiter in dieser Funktion nicht lange auf Ergebnisse warten können. Schnelle und effiziente Datenabfragen ermöglichen dem Supportteam, Anfragen und Probleme zeitnah zu bearbeiten. Dies steigert die Kundenzufriedenheit und verbessert die Effektivität der Supportprozesse.

In einem Supportumfeld ist der Zugriff auf aktuelle und präzise Informationen entscheidend, um Kundenanliegen effizient zu lösen. Lange Wartezeiten bei Datenabfragen können nicht nur zu Verzögerungen in der Bearbeitung von Kundenanfragen führen, sondern auch die Qualität der Kundenerfahrung beeinträchtigen.

Die Performance der Datenabfragen beeinflusst somit direkt die Reaktionszeit des Supportteams. Schnelle Zugriffe auf relevante Daten ermöglichen den Mitarbeitern, fundierte Entscheidungen zu treffen, um Kundenprobleme rasch zu identifizieren und Lösungen anzubieten. Eine effiziente Datenabfrage-Performance trägt somit maßgeblich zur Optimierung der Supportprozesse bei und fördert eine zeitnahe, kundenorientierte Problemlösung.

\subsection{Problemstellung}
\setauthor{Felix Arzt}
Trotz Optimierungsversuchen verbleibt für einige Abfragen eine signifikante Dauer von über einer Minute. Das liegt daran, dass bestimmte Abfragen mehrere Datenbankzugriffe, JOIN-Operationen und Gruppierungen involvieren. Da eine derartige Zeitspanne für den Support inakzeptabel ist, sind zeitintensive Abfragen \textbf{Cron-Jobs} ausgelagert, um die Auswirkung auf die unmittelbare Support-Interaktion zu minimieren.

\subsection{Cron-Jobs}
\setauthor{Felix Arzt}
Cron-Jobs sind automatisierte Aufgaben den in unixartigen Betriebssystemen, wie Linux, MacOS und Serverumgebungen ausgeführt werden können. Es gibt den sogenannten Cron-Daemon, welcher die Grundlagen für die Cron-Jobs bildet und im Hintergrund mitläuft. Dieser gibt die zeitlichen Impulse für die Aufgaben und führt diese zur angegebenen Zeit aus. Die Syntax, um einen Cron-Job an einer bestimmten Uhrzeit zu starten ist wie folgt:
\cite{cron_jobs_basics}
\newline
\begin{lstlisting}
'* * * * *', <auszufuehrender Befehl>
\end{lstlisting}
Dabei steht der fünfte Stern für den Wochentag, an dem der Befehl ausgeführt werden soll. In diesem Fall sind Werte von 0 bis 7 möglich, dabei steht 0 und 7 für den Sonntag.
\newline
Der vierte Stern steht für den Monat, hierbei sind anstelle des Stern Zahlen von 1 bis 12 möglich.
\newline
Der dritte Stern steht für den Wochentag und es sind Zahlen von 1 bis 31 möglich. Wählt man die Zahl 31 wird dieser Befehl nur in Monaten ausgeführt, die 31 Tage dauern, es wird dementsprechend kein Zeitintervall gestartet und alle 31 Tage ausgeführt, sondern es wird tatsächlich vom Cron-Daemon der Tag überprüft.
\newline
Der zweite Stern gibt an zu welcher Stunde der Befehl ausgeführt werden soll. Dabei sind Zahlen von 0 bis 23 möglich.
\newline
Der erste Stern gibt die Minuten an und kann Zahlen von 0 bis 59 annehmen.
\newline
Es ist ebenfalls möglich die Sterne nicht durch Zahlen zu ersetzen, dies würde bedeuten, dass der Befehl jede Minute, jede Stunde, jeden Tag, jeden Monat oder jeden Wochentag ausgeführt werden würde. Möchte man den Befehl zum Beispiel jeden Tag um 8 und um 20 Uhr ausführen würde der Befehl wie folgt aussehen:
\newline
\begin{lstlisting}
'0 8,20 * * *', <auszufuehrender Befehl>
\end{lstlisting}
Möchte man zum Beispiel einen Befehl alle 10 Minuten ausführen, muss man nicht wie im obigen Beispiel alle Minuten manuell eingeben, sondern kann dies wie folgt realisieren:
\newline
\begin{lstlisting}
'*/10 * * * *', <auszufuehrender Befehl>
\end{lstlisting}
\cite{cron_jobs_scheduling_examples}

\subsection{Lösung durch Cron-Jobs}
\setauthor{Felix Arzt}
\input{content/database_performance/solution_through_cron_jobs}

\subsection{Implementierung}
\setauthor{Felix Arzt}
Um Cron-Jobs in diesem express.js Projekt zu implementieren, muss man das NPM-Package \textbf{node-cron} installieren.
\begin{verbatim}
npm install node-cron
\end{verbatim}
Nach Abschluss der Installation wird das Cron-Modul in ein eigens erstelltes JavaScript File importiert.
\begin{verbatim}
const cron = require('node-cron')
\end{verbatim}
Mit der oben erstellen cron-Variable ist es nun möglich, jegliche Cron-Jobs einzuplanen.
\newline
\begin{lstlisting}
cron.schedule('0 22 * * *', <auszufuehrender Befehl>)
\end{lstlisting}
Dabei wird in dieser Diplomarbeit ebenfalls noch ein if-Statement eingebaut, welches mittels .env Variablen überprüft, ob der Server gerade im sogenannten "production" Modus läuft. Falls das Backend lokal gestartet wird, werden alle Cron-Jobs erneut ausgeführt, unabhängig von der eingeplanten Zeit. Im folgenden Codebeispiel sieht man die oben beschriebenen Funktionen.
\newline
\begin{lstlisting}[caption=Einplanen der Cron-Jobs]
const cronJobs = async () => {
  // on localhost, run cronjobs immediately after startup
  if (process.env.NODE_ENV !== 'production' && !process.env.MONGODB_CONNECTION_STRING) {
    await estimateUsersWithExceededStorage()
    await estimateUsersWithoutProject()
    await estimateStatistics()
    await estimateProjectOwners()
    await estimateSecretUsers()
  }

  logger.info('Cron Jobs scheduled')
  cron.schedule('0 22 * * *', estimateUsersWithExceededStorage)
  cron.schedule('0 22 * * *', estimateStatistics)
  cron.schedule('50 21 * * *', estimateProjectOwners)
  cron.schedule('0 22 * * *', estimateSecretUsers)
}

module.exports.cronJobs = cronJobs
\end{lstlisting}
Die gesamte Funktion mit der Funktionalität wird schlussendlich exportiert und im server.js File wieder importiert. In diesem wird die Funktion nach dem erfolgreichen Verbinden zu den beiden Datenbanken aufgerufen und dadurch eingeplant.
